\documentclass[12pt]{article}
\usepackage{graphicx}
\usepackage{amsmath, amssymb, amsthm}
\usepackage[auth-sc,affil-sl]{authblk}
\usepackage{enumerate}
\newtheorem{thm}{Theorem}
\newtheorem{cor}[thm]{Corollary}
\newtheorem{lem}[thm]{Lemma}
\newtheorem*{claim}{Claim}
\newtheorem*{mydef}{Definition}

% Set the margins
%
\setlength{\textheight}{8.5in}
\setlength{\headheight}{.25in}
\setlength{\headsep}{.25in}
\setlength{\topmargin}{0in}
\setlength{\textwidth}{6.5in}
\setlength{\oddsidemargin}{0in}
\setlength{\evensidemargin}{0in}

% Macros
\newcommand{\myN}{\hbox{N\hspace*{-.9em}I\hspace*{.4em}}}
\newcommand{\myZ}{\hbox{Z}^+}
\newcommand{\myR}{\hbox{R}}

\newcommand{\myfunction}[3]
{${#1} : {#2} \rightarrow {#3}$ }

\newcommand{\myzrfunction}[1]
{\myfunction{#1}{{\myZ}}{{\myR}}}


% Formatting Macros

\newcommand{\myheader}[4]
{\vspace*{-0.5in}
\noindent
{#1} \hfill {#3}

\noindent
{#2} \hfill {#4}

\noindent
\rule[8pt]{\textwidth}{1pt}

\vspace{1ex} 
}  % end \myheader 

\newcommand{\myalgsheader}[0]
{\myheader
{ {\bf{COS 432}} }
{ {\bf{Fall 2014}} }
{ {\bf{Charles Marsh (crmarsh@)}} }
{ {\bf{Lecture 13: Voting}} }
}

\begin{document}

\myalgsheader

\pagestyle{plain}

\section*{Voting Criteria}

Requirements for electronic voting include:
\begin{itemize}
\item Only authorized voters can vote.
\item $\leq 1$ ballot for each voter.
\item Ballots counted as cast.
\item Ballots are \textbf{secret} or ``receipt-free": there's no way to prove to a third party that you voted a particular way (to prevent voter coercion).
\end{itemize}

There are logically two steps:
\begin{enumerate}
\item Cast ballot in a ``ballot box".
\item Tally ``ballot box" to get result.
\end{enumerate}

\section*{Paper Ballots}

If you think about old-fashioned paper ballots, they have the following properties:
\begin{itemize}
\item Cheap to operate.
\item Easy to understand.
\item Problematic if the ballot is long or complex.
\item Subject to trickery, e.g., with chain-voting, where a goon hands you a filled-out ballot and demands that you bring out an empty ballot.
\item Chain of custody: what happens with the ballot box between the time votes are cast and counted?
\end{itemize}

\section*{End-to-End (E2E) Cryptographic Voting}

The basic outline is as follows:
\begin{enumerate}
\item Voter encrypts their ballot.
\item Voter casts the ciphertext vote.
\item System publishes the ciphertext ballots on a public bulletin board.
\item End of election tally:
\begin{enumerate}
\item Shuffle and re-encrypt ciphertext ballots.
\item Trustees collectively decrypt the ciphertext ballots, producing plaintext ballots.
\item Publish everything, so anyone can do the tally.
\end{enumerate}
\end{enumerate}

``Re-encryption" implies randomized encryption: one plaintext maps to many possible ciphertexts. This is an operation that, given a ciphertext $C$, computes $reenc(C)$ using a randomized algorithm with the property that $decrypt(reenc(C)) = decrypt(C)$. Essentially, the re-encryption still allows you to decrypt to the original plaintext.

\subsection*{El Gamal Encryption}

For El Gamal encryption, you have public parameters $g, p$ (Diffie-Hellman) and a private key $x$. The public key is computed as $g^x \bmod{p}$.

To encrypt a message $M$:
\begin{enumerate}
\item Pick a random $r$.
\item Compute $(g^r \bmod{p}, m g^{rx} \bmod{p})$. Notice that this is computable knowing only the public key.
\end{enumerate}

To decrypt a ciphertext $(A, B)$, compute $A^{-x}B \bmod{p}$. Notice that this is only computable if you know the private key $x$.

\textbf{El Gamal allows for re-encryption}: given $(A, B)$, you can generate a new random value $r'$ and compute:

\begin{align*}
(A g^{r'} \bmod{p}, B g^{r'x} \bmod{p}) &= (g^r g^{r'} \bmod{p}, m g^{rx} g^{r'x} \bmod{p}) \\
&= (g^{r+r'} \bmod{p}, m g^{(r+r')x} \bmod{p})
\end{align*}

Notice that the second line is just El Gamal encryption with random value $(r + r')$, i.e., it is a valid re-encryption, as it will decrypt to the same plaintext.

As an added bonus, you can prove to challengers that you performed a valid re-encryption by revealing the value $r'$.

\subsection*{Shuffling}

We start with a ``ballot box" $B$ and end with $B'$, which should be a reordering of $B$. To prove that the shuffler didn't cheat, we'll need to use a (zero-knowledge) proof protocol:

\begin{enumerate}
\item Prover (shuffler) produces $B_1$. $B_1$ should be equivalent to $B$ and $B'$.
\item Prover knows the correspondence.
\item If $B$ is \textit{not} equivalent to $B'$, then $B_1$ cannot be equivalent to both of them.
\item To challenge the prover, we have the challenger flip a coin and ask the prover to show the equivalence between $B$ and $B_1$ or $B'$ and $B_1$ (depending on the result of the coin flip).
\item Prover unwraps the equivalence by demonstrating the correspondence: ``You can get from $B'$ to $B$ by a re-encryption with random value $r$."
\item The prover will fail at least half the time if he or she doesn't actually have a valid correspondence.
\item To increase our guarantee, run the proof protocol over and over.
\end{enumerate}

\section*{Lower-Tech Voting}

\begin{center}
    \begin{tabular}{| l | l | l |}
    \hline
    Type of record & Paper & Electronic \\ \hline
    Example & Paper-ballots in ballot box & Voting machines \\ \hline
    Counting & Slow, expensive & Fast, cheap \\ \hline
    Main threat & Tampering afterward & Tampering beforehand \\ \hline
    \end{tabular}
\end{center}

\subsection*{Paper with Electronic Records}

As an example, consider optical-scan voting:
\begin{itemize}
\item The voters fills out a paper ballot.
\item The voter feeds the ballet into a scanner.
\item The scanner records an electronic record.
\item The paper ballot is fed into a ballot box.
\end{itemize}

This produces both a paper and electronic ballot.

\end{document}
